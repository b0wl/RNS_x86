%% This is a skeleton file demonstrating the advanced use of IEEEtran.cls
%% (requires IEEEtran.cls version 1.8b or later) with an IEEE Computer
%% Society journal paper.

\documentclass[10pt,journal,compsoc]{IEEEtran}

\usepackage[nocompress]{cite}

\usepackage[pdftex]{graphicx}
\graphicspath{{./jpeg/}}
\DeclareGraphicsExtensions{.pdf,.jpeg,.png}

\usepackage{amsmath}
\usepackage[listings,skins]{tcolorbox}
\interdisplaylinepenalty=2500

\newcommand\MYhyperrefoptions{bookmarks=true,bookmarksnumbered=true,
pdfpagemode={UseOutlines},plainpages=false,pdfpagelabels=true,
colorlinks=true,linkcolor={black},citecolor={black},urlcolor={black},
pdftitle={Rozszerzenie Programowe RNS Procesora x86 Mnożenie Montgomerego},%<!CHANGE!
pdfsubject={Rozszerzenie Programowe RNS Procesora x86 Mnożenie Montgomerego},%<!CHANGE!
pdfauthor={J. Jurec F. Toruń},%<!CHANGE!
pdfkeywords={System Resztowy, Chińskie Twierdzenie o Resztach, Systemy Liczbowe, Arytmetyka Komputerowa, Organizacja i Architektura Komputerów, Piotr Patronik Projekt, Mnożenie Montgomerego, Kryptografia, Optymalizacja }}%<^!CHANGE!

\hyphenation{op-tical net-works semi-conduc-tor}

\usepackage{polski}
\usepackage[utf8]{inputenc}
\renewcommand\IEEEkeywordsname{Słowa kluczowe}

\begin{document}
\title{Rozszerzenie Programowe RNS Procesora x86 \\ Mnożenie Montgomerego}

\author{Jan~Jakub~Jurec,~\IEEEmembership{Student,~PWR,}
        Filip~Toruń,~\IEEEmembership{Student,~PWR}}

\markboth{sprawozdanie z projektu z organizacji i architektury komputerów, Czerwiec~2017}%
{Shell \MakeLowercase{\textit{et al.}}: sprawozdanie z projektu z organizacji i architektury komputerów, Czerwiec~2017}
% The only time the second header will appear is for the odd numbered pages
% after the title page when using the twoside option.

% Justowanie abstraktu
\makeatletter
\long\def\@IEEEtitleabstractindextextbox#1{\parbox{0.922\textwidth}{#1}}
\makeatother

\IEEEtitleabstractindextext{
\begin{abstract}
Kryptografia jest jedną z najszybciej rozwijających się i najbardziej kluczowych dziedzin informatyki. Skuteczne zaszyfrowanie danych pozwala na bezpieczne ich przechowywanie oraz wymianę choćby w systemach bankowych czy prywatnych rozmowach. Do najbezpieczniejszych systemów kryptograficznych należą kryptosystemy asymetryczne, wśród których bardzo popularne są algorytmy RSA i Diffiego-Hellmana. Oba oparte są na operacjach modularnej na wielkich liczbach, która jest niezwykle kosztowna. Ten artykuł pokazuje drogę, jaką przeszli autorzy, by zbliżyć się do zrozumienia  działania mnożenia~Montgomerego - algorytmu, który znacznie zmniejsza koszt obliczeń w arytmetyce modulo. 
\end{abstract}

\begin{IEEEkeywords}
System Resztowy, Chińskie Twierdzenie o Resztach, Systemy Liczbowe, Arytmetyka Komputerowa, Organizacja i Architektura Komputerów, Piotr Patronik Projekt, Mnożenie Montgomerego, Kryptografia, Optymalizacja 
\end{IEEEkeywords}}


\maketitle


\IEEEdisplaynontitleabstractindextext
\IEEEpeerreviewmaketitle


\IEEEraisesectionheading{\section{WSTĘP}\label{sec:wstęp}}
\IEEEPARstart{A}{by} zrozumieć działanie mnożenia Montgomerego należy najpierw poznać własności systemu resztowego i zrozumieć, jak przebiegają w nim operacje. W tym celu autorzy zaimplementowali konwersję w przód i w tył, używając chińskiego twierdzenia o resztach. Później napisali również dodawanie, odejmowanie, mnożenie oraz dzielenie przez potęgę liczby 2 w systemie resztowym dla liczby 32-bitowej w systemie modułów 7, 15, 31, 127, 8192. Rezultat tych zmagań, które zajęły większą część czasu projektu udostępniają autorzy w Dodatku A. Po zapoznaniu się z arytmetyką modularną autorzy przystąpili do implementacji algorytmu mnożenia Montgomerego w samodeskryptywnym języku skryptowym Python. Wynikowy kod udostępniają w Dodatku B. Własności mnożenia w systemie resztowym zauważone przez Piotra Montgomerego pozwalają na drastyczne przyspieszenie multiplikacji a tym samym podnoszenia do potęgi w tymże systemie. Dodatkowo, odpowiednio dobierając parametry systemu modulo, można jeszcze bardziej usprawnić obliczenia wykorzystując natychmiastowość dzielenia oraz uzyskiwania reszty z dzielenia przez potęgi liczby 2 w komputerach ogólnego zastosowania.

\section{MNOŻENIE MONTGOMEREGO}
\IEEEPARstart{W}{arty} ponownego podkreślenia jest fakt, że produktem mnożenia Montgomerego jest liczba w systemie modulo n. Wszystkie operacje przedstawione w tym rozdziale odbywają się właśnie w takim systemie. Należy porzucić intuicje wyniesione z arytmetyki liczb wymiernych.\\
Algorytm Montgomerego działa dla dowolnie wielkich liczb naturalnych. Operuje on bowiem na słowach liczby. Dla przykładu: liczba 256-bitowa może składać się z 8 słów 32-bitowych albo 4 64-bitowych. Wynikiem mnożenia Montgomerego jest produkt Montgomerego o następującym zapisie:

\begin{align*}
  &MonPro(a,b) = abr^{-1}\ mod\ n
\end{align*}
\noindent
Zdaniem autorów przed objaśnieniem wprowadzonych symbolów warto wyjaśnić, że $r^{-1}$ nie oznacza odwrotności liczby w sensie wymiernym a raczej liczbę, która po przemnożeniu przez $r$ da resztę równą $1$ po podzieleniu przez $n$. Jest to odwrotność multiplikatywna. W ścisłym zapisie matematycznym:

\begin{align*}
  &rr^{-1}=1\ (mod\ n)
\end{align*}
\noindent
Wprowadzona liczba $n$ oznacza podstawę systemu resztowego a $r$ arbitralnie jest dobranym dodatkowym parametrem. Ważne, żeby $a$, $b$, $r$ i $n$ spełniały poniższe dodatkowe założenia:  

\begin{align*}
   &a, b < n  \\
   &nwd(n,r) = 1  \\
   &ab < r 
\end{align*}
\noindent
Aby dodatkowo usprawnić działanie algorytmu założono, że:

\begin{align*}
  &r = 2^k \\
  &2^{k-1} \leq n < 2^k 
\end{align*}
\noindent
Spełnienie pierwszego równania zapewni szybkość wykonania operacji z użyciem $r$, drugiego natomiast spowoduje, że możliwe będzie używanie względnie dużych $a$ i $b$. Przed wprowadzeniem nowych symboli zostanie wreszcie przedstawiony algorytm mnożenia Montgomerego w trzech krokach. 

\begin{align*}
  &t = ab\\
  &u = \frac{t+(tn'\ mod\ r)n}{r}\\
  &u \geq n\ ?\ r = u-n\ :\ r = u
\end{align*}
\noindent
Jasnym jest, że obliczenie $t$ w kroku pierwszym przyspieszy późniejsze obliczenia. Krok drugi natomiast praktycznie oblicza produkt z $r^{-1}$ w $(mod\ n)$. Ostatni krok tak naprawdę wykonuje działanie oblicza już ostatnią resztę z dzielenia przez $n$ poprzez odjęcia $n$ od ewentualnie za dużego wyniku. Nowym wprowadzonym do działań symbolem jest $n'$. Jest to liczba spełniająca równanie:

\begin{align*}
  &rr^{-1}-nn'=1\ (mod\ n)
\end{align*}

a więc:

\begin{align*}
  &n'= \frac{1}{r-n}\ mod\ n
\end{align*}

\noindent
Oznacza to tyle, że $n'$ jest odwrotnością multiplikatywną liczby $r-n$ w $(mod\ n)$. \\ \newline
Przedstawione implementacje mnożenia Montgomerego korzystają z powyższych kroków. Metoda Oddzielnego Skanowania Operandów (en. SOS - Separated Operand Scanning) wszystkie kroki wykonuje kaskadowo, po sobie. Jednak metoda Zgrubnie Zintegrowanego Skanowania Operandów (en. CIOS - Coarsely Integrated Operand Scanning) łączy pierwsze dwa kroki, dzięki czemu zajmuje mniej miejsca w pamięci oraz wymaga wykonania mniejszej ilości instrukcji. W żadnej implementacji nie jest użyte $r^{-1}$. Jest ono jednak potrzebne do sprawdzenia poprawności wyniku na przykład w programie \texttt{gp}.

\section{IMPLEMENTACJA}
\subsection{Metoda Separated Operand Scanning}\label{sec:metoda1}
\IEEEPARstart{P}{ierwszą} metodą analizowaną przez autorów była metoda Separated Operand Scanning. Postępuje ona zgodnie z przedstawionym powyżej schematem i dzieli się na 3 kroki.
Pierwszy krok wygląda następująco:
\begin{lstlisting}
for i=0 to s-1
  C := 0
    for j=0 to s-1
      (C,S) := t[i+j] + a[j]*b[i] + C
      t[i+j] := S
    t[i+s] := C
\end{lstlisting}

\noindent Przy czym dodać trzeba, że tablica t musi być wyzerowana, będzie mieć ona długość 2s słów.\\ Krok drugi: 
\begin{lstlisting}
for i=0 to s-1
  C := 0
  m := t[i]*n'[0] mod W
  for j=0 to s-1
    (C,S) := t[i+j] + m*n[j] + C
    t[i+j] := S
  ADD (t[i+s],C)
for j=0 to s
  u[j] := t[j+s]
\end{lstlisting}

\noindent Tutaj produktem jest tablica $u$ o długości s+1 słów. Wartym nadmienienia jest, że wykorzystany tu został $n'[0]=n'\ mod\ 2^w$ $-$ - najmniej znaczący bit odwrotności multiplikatywnej $z~r-n$. Funkcja ADD propaguje zadane przeniesienie na bardziej znaczące słowa.

\noindent Ostatni krok powtarza się w obu omawianych metodach. I jest to zwykłe porównanie $u$ z $n$ i zwrócenie $u-n$ gdy $u$ jest większe, a w przeciwnym wypadku zwrócenie po prostu $u$.
\begin{lstlisting}
B := 0
for i=0 to s-1
  (B,D) := u[i] - n[i] - B
  t[i] := D
(B,D) := u[s] - B
t[s] := D
if B = 0 then return t[0],t[1],...t[s-1]
	 else return u[0],u[1],...u[s-1]
\end{lstlisting} 
\noindent Ten krok zwraca już wynik.

\subsection{Metoda Coarsely Integrated Operand Scanning}\label{sec:metoda2}
\IEEEPARstart{D}{ruga} metoda integruje krok pierwszy oraz drugi, dzięki czemu zmniejszone jest zapotrzebowanie na pamięć. Integracja ta polega na wykorzystaniu zewnętrznej pętli do obu operacji - mnożenia oraz redukcji. Jest to możliwe ponieważ $m[i]$ zależne jest jedynie od $t[i]$.
\begin{lstlisting}
for i = 0 to s - 1
  C := 0
  for j = 0 to s - 1
    (C,S) := t[j] + a[j]b[i] + C
    t[j] := S
  (C,S) := t[s] + C
  t[s] := S
  t[s + 1] := C
  C := 0
  m := t[0]n'[0] mod W
  for j = 0 to s - 1
    (C,S) := t[j] + mn[j] + C
    t[j] := S 
    (C,S) := t[s] + C
    t[s] := S
    t[s + 1] := t[s + 1] + C
    for j = 0 to s
      t[j] := t[j + 1]
\end{lstlisting} 

\noindent Znowu tablica $t$ musi być wyzerowana na starcie. Tutaj można już zastosować krok 3 na tablicy t by otrzymać wynik.

\section{PODSUMOWANIE}
\IEEEPARstart{A}{utorzy} są zdania, że zrealizowali cel projektu. Poprawnie zaimplementowali dwa algorytmy mnożenia Montgomerego (w tym jeden bardzo wydajny czasowo i pamięciowo) dla liczb składających się z dowolnej ilości słów dowolnej wielkości. Zrozumiawszy sztuczki matematyczne użyte w przykładowych implementacjach dokonanych przez Koca i Acara rozsądnym czasie zdołaliby zaimplementować kolejne algorytmy.  \\
Dobrym pomysłem było podzielenie projektu na dwie części. Łagodne wprowadzenie do systemów resztowych zrealizowane poprzez implementację prostych algorytmów dało autorom podstawowe obycie i intuicję podczas analizowania algorytmów przedstawionych przez ww. badaczy. \\
Jest pocieszającym fakt, że dzięki pokazanym algorytmom można znacznie przyspieszyć mnożenie a przez to podnoszenie do potęgi w systemach modulo nawet na niededykowanym do tego sprzęcie elektronicznym - na przykład komputerach osobistych. Dzięki naukowcom takim jak Piotr Montgomery kryptografia i bezpieczeństwo danych nie są domeną wybranych, których stać na specjalistyczny hardware a stają się dostępne dla wszystkich. Bez wątpienia w konsekwencji zwiększa to spokój wewnętrzny i komfort ogółu zinformatyzowanego świata.

\appendices
\section{Podstawowe operacje RNS w ASM~AT\&T}
\begin{lstlisting}
.data
EXIT_SUCCESS=0
SYSEXIT=60

value_rns:
# 123456  100 0110 01110 0001100 0001001000000
    .quad 0x8ce18240
value_pos:
    .quad 123456

# 2^3  - 1 (for relative primarity)
# 2^4  - 1
# 2^5  - 1
# 2^7  - 1
# 2^13 - 1
m1: .quad  7         
m2: .quad  15        
m3: .quad  31        
m4: .quad  127       
m5: .quad  8192      

# 7 * 15 * 31 * 127 * 8192
M:  .quad 3386449920 

# 3386449920 / 7
# 3386449920 / 15
# 3386449920 / 31
# 3386449920 / 127
# 3386449920 / 8192
M1: .quad 483778560  
M2: .quad 225763328  
M3: .quad 109240320  
M4: .quad 26664960   
M5: .quad 413385     

#  multiplicative inversions
y1: .quad 6          
y2: .quad 2
y3: .quad 7
y4: .quad 54
y5: .quad 2937

#  Convert positional system number 
#  to RNS number
#  RAX(rns) = ARG(pos)
.macro rns pos_num
  push %rbx
  push %rcx
  push %rbx
  push %r9

  mov \pos_num, %r9
  xor %rbx, %rbx      

  mov %r9, %rax
  xor %rdx, %rdx
  mov $7, %rcx
  div %rcx
  shl $29, %rdx
  mov %rdx, %rbx

  mov %r9, %rax
  xor %rdx, %rdx
  mov $15, %rcx
  div %rcx
  shl $25, %rdx
  or  %rdx, %rbx

  mov %r9, %rax
  xor %rdx, %rdx
  mov $31, %rcx
  div %rcx
  shl $20, %rdx
  or  %rdx, %rbx

  mov %r9, %rax
  xor %rdx, %rdx
  mov $127, %rcx
  div %rcx
  shl $13, %rdx
  or  %rdx, %rbx

  mov %r9, %rax
  xor %rdx, %rdx
  mov $8192, %rcx
  div %rcx
  or  %rdx, %rbx

  mov %rbx, %rax

  pop %r9
  pop %rbx
  pop %rcx
  pop %rdx
.endm

#  Convert RNS number to positional 
#  system number
#  RAX(pos) = ARG(rns)
.macro drns rns_num
  push %r11
  push %r8
  push %r9
  push %rbx
  push %rdx
  push %rcx

  mov \rns_num, %r9

  mov %r9, %rax
  shr $29, %rax   
  and $7, %rax    
  mov %rax, %r11  


  mov M1, %rax
  mul %r11
  mov y1, %rbx
  mul %rbx
  mov %rax, %r8

  mov %r9, %rax
  shr $25, %rax   
  and $15, %rax   
  mov %rax, %r11  


  mov M2, %rax
  mul %r11
  mov y2, %rbx
  mul %rbx
  add %rax, %r8

  mov %r9, %rax
  shr $20, %rax   
  and $31, %rax   
  mov %rax, %r11  


  mov M3, %rax
  mul %r11
  mov y3, %rbx
  mul %rbx
  add %rax, %r8

  mov %r9, %rax
  shr $13, %rax   
  and $127, %rax  
  mov %rax, %r11  


  mov M4, %rax
  mul %r11
  mov y4, %rbx
  mul %rbx
  add %rax, %r8

  mov %r9, %rax
                
  and $8191, %rax
  mov %rax, %r11 


  mov M5, %rax
  mul %r11
  mov y5, %rbx
  mul %rbx
  add %rax, %r8


  mov M, %rbx
  xor %rdx, %rdx
  mov %r8, %rax
  div %rbx
  mov %rdx, %rax

  pop %rcx
  pop %rdx
  pop %rbx
  pop %r9
  pop %r8
  pop %r11
.endm

#  Add two RNS numbers. 
#  One in RAX, other as ARG.
#  RAX = RAX + ARG
.macro addrns rns_num
  push %rbx
  push %rcx
  push %rdx
  push %r9
  push %r10
  push %r11
  push %r12

  mov \rns_num, %r12
  mov %rax, %r9
  mov %r12, %rax
  xor %r11, %r11
  shr $29, %rax  
  and $7, %rax   
  mov %rax, %r10
  mov %r9, %rax
  shr $29, %rax  
  and $7, %rax   
  add %r10, %rax
  mov $7, %rbx
  xor %rdx, %rdx
  div %rbx
  shl $29, %rdx
  or %rdx, %r11

  mov %r12, %rax
  shr $25, %rax  
  and $15, %rax  
  mov %rax, %r10
  mov %r9, %rax
  shr $25, %rax  
  and $15, %rax  
  add %r10, %rax
  mov $15, %rbx
  xor %rdx, %rdx
  div %rbx
  shl $25, %rdx
  or %rdx, %r11

  mov %r12, %rax
  shr $20, %rax    
  and $31, %rax    
  mov %rax, %r10
  mov %r9, %rax
  shr $20, %rax    
  and $31, %rax    
  add %r10, %rax
  mov $31, %rbx
  xor %rdx, %rdx
  div %rbx
  shl $20, %rdx
  or %rdx, %r11

  mov %r12, %rax
  shr $13, %rax     
  and $127, %rax    
  mov %rax, %r10
  mov %r9, %rax
  shr $13, %rax     
  and $127, %rax    
  add %r10, %rax
  mov $127, %rbx
  xor %rdx, %rdx
  div %rbx
  shl $13, %rdx
  or %rdx, %r11

  mov %r12, %rax    
  and $8191, %rax   
  mov %rax, %r10
  mov %r9, %rax
  and $8191, %rax   
  add %r10, %rax
  mov $8192, %rbx
  xor %rdx, %rdx
  div %rbx
  or %rdx, %r11

  mov %r11, %rax

  pop %r12
  pop %r11
  pop %r10
  pop %r9
  pop %rdx
  pop %rcx
  pop %rbx
.endm

#  Multiple two RNS numbers. 
#  One in RAX, other as ARG.
#  RAX = RAX * ARG
.macro mulrns rns_num
  push %rbx
  push %rcx
  push %rdx
  push %r9
  push %r10
  push %r11
  push %r12

  mov \rns_num, %r12
  mov %rax, %r9
  mov %r12, %rax
  xor %r11, %r11
  shr $29, %rax     
  and $7, %rax      
  mov %rax, %r10
  mov %r9, %rax
  shr $29, %rax     
  and $7, %rax      
  mul %r10
  mov $7, %rbx
  xor %rdx, %rdx
  div %rbx
  shl $29, %rdx
  or %rdx, %r11

  mov %r12, %rax
  shr $25, %rax     
  and $15, %rax     
  mov %rax, %r10
  mov %r9, %rax
  shr $25, %rax     
  and $15, %rax     
  mul %r10
  mov $15, %rbx
  xor %rdx, %rdx
  div %rbx
  shl $25, %rdx
  or %rdx, %r11

  mov %r12, %rax
  shr $20, %rax     
  and $31, %rax     
  mov %rax, %r10
  mov %r9, %rax
  shr $20, %rax     
  and $31, %rax     
  mul %r10
  mov $31, %rbx
  xor %rdx, %rdx
  div %rbx
  shl $20, %rdx
  or %rdx, %r11

  mov %r12, %rax
  shr $13, %rax     
  and $127, %rax    
  mov %rax, %r10
  mov %r9, %rax
  shr $13, %rax     
  and $127, %rax    
  mul %r10
  mov $127, %rbx
  xor %rdx, %rdx
  div %rbx
  shl $13, %rdx
  or %rdx, %r11

  mov %r12, %rax    
  and $8191, %rax   
  mov %rax, %r10
  mov %r9, %rax
  and $8191, %rax   
  mul %r10
  mov $8192, %rbx
  xor %rdx, %rdx
  div %rbx
  or %rdx, %r11

  mov %r11, %rax

  pop %r12
  pop %r11
  pop %r10
  pop %r9
  pop %rdx
  pop %rcx
  pop %rbx
.endm

.macro shr_rns_step
  push %rbx
  push %rcx
  push %rdx
  push %r9
  push %r10
  push %r11

  xor %rdx, %rdx
  xor %r11, %r11

  mov %rax, %r9
  mov %rax, %r8

  shr $29, %rax    
  and $7, %rax     
  mov %rax, %r10
  shr %r10         
  and $1, %rax     
  shl $2, %rax     
  or %r10, %rax
  shl $29, %rax
  or %rax, %r11

  mov %r9, %rax

  shr $25, %rax    
  and $15, %rax    
  mov %rax, %r10
  shr %r10         
  and $1, %rax     
  shl $3, %rax     
  or %r10, %rax
  shl $25, %rax
  or %rax, %r11

  mov %r9, %rax

  shr $20, %rax    
  and $31, %rax    
  mov %rax, %r10
  shr %r10         
  and $1, %rax     
  shl $4, %rax     
  or %r10, %rax
  shl $20, %rax
  or %rax, %r11

  mov %r9, %rax

  shr $13, %rax   
  and $127, %rax  
  mov %rax, %r10
  shr %r10        
  and $1, %rax    
  shl $6, %rax    
  or %r10, %rax
  shl $13, %rax
  or %rax, %r11

  mov %r9, %rax
                
  and $8191, %rax 
  mov %rax, %r10
  shr %r10        
  and $1, %rax    
  shl $12, %rax   
  or %r10, %rax
  or %rax, %r11

  mov %r11, %rax

  pop %r11
  pop %r10
  pop %r9
  pop %rdx
  pop %rcx
  pop %rbx
.endm

.macro shr_rns positions
  push %rsi
  mov \positions, %rsi

filip_tribute:
  cmp $0, %rsi
  jle exit_shr_rns

  shr_rns_step
  dec %rsi
  jmp filip_tribute

exit_shr_rns:
  pop %rsi
.endm

#  Compare two RNS numbers. 
#  One in RAX, other as ARG.
#  If RAX bigger -> RAX = 1,
#  If RAX smaller -> RAX = -1,
#  If equal -> RAX = 0
.macro cmprns rns_num
  push %rbx

  drns %rax
  mov %rax, %rbx
  mov \rns_num, %rax
  drns %rax
  cmp %rax, %rbx
  jl arg_greater
  je both_equal

rax_greater:
  mov $1, %rax
  jmp leave_cmprns

arg_greater:
  mov $-1, %rax
  jmp leave_cmprns

both_equal:
  xor %rax, %rax

leave_cmprns:
  pop %rbx
.endm

.text
.global main
main:
  movq %rsp, %rbp

rns_check:
  rns $128

  shr_rns $2

  drns %rax

exit:
  movq $SYSEXIT, %rax
  movq $EXIT_SUCCESS, %rdi
  syscall

\end{lstlisting}

\section{Mnożenie Montgomerego SOS i CIOS w Pythonie}
\begin{lstlisting}
#return array of base-system words
def radix(x, base):
    digits = []

    while x:
        digits.append(int(x % base))
        x /= int(base)

    digits.reverse()
    return digits

# ax + by = gcd(a, b)
def egcd(a, b):
    x0, x1, y0, y1 = 1, 0, 0, 1
    while b != 0:
        q, a, b = a // b, b, a % b
        x0, x1 = x1, x0 - q * x1
        y0, y1 = y1, y0 - q * y1
    return a, x0, y0


# Multiplicative inverse 1/d mod n
def mulinv(d, n):
    g, x, _ = egcd(d, n)
    if g == 1:
        return x % n

# Propagate carry
def ADD(t, i, C, W):
    for j in range(i, len(t)):
        if C:
            d_w = t[j] + C
            t[j] = d_w % W
            C = d_w // W
        else:
            return t

#Mon. Mul. Separate Operand Scanning
def MonProSOS(a, b, s, w, n):
    k = s * w
    W = 2 ** w
    r = 2 ** k
    n_p = mulinv(r-n, r)
    n0 = n_p % W
    aT = radix(a, W)[-s:][::-1]
    bT = radix(b, W)[-s:][::-1]
    nT = radix(n, W)[-s:][::-1]

    # Step 1
    t = [0]*(2*s)
    for i in range(s):
        C = 0
        for j in range(s):
            d_w = t[i + j] + aT[j] * bT[i] + C
            C, S = d_w // W, d_w % W
            t[i + j] = S
        t[i + s] = C

    # Step 2
    t = t + [0]
    C2 = 0
    for i in range(s):
        C = 0
        m = (t[i] * n0) % W
        for j in range(s):
            d_w = t[i + j] + m * nT[j] + C
            C, S = d_w // W, d_w % W
            t[i + j] = S

        t = ADD(t, i + s, C, W)
    u = t[s:]

    # Step 3
    B = 0
    for i in range(s):
        d_w = u[i] - nT[i] - B
        B, D = d_w // W, d_w % W
        t[i] = D
    d_w = u[s] - B
    B, D = d_w // W, d_w % W
    t[s] = D
    if B:
        cT = t[:s]
    else:
        cT = u[:s]
    c = 0
    for i, w in enumerate(cT):
        c += w*W**i
    return c


def MonProCIOS(a, b, s, w, n):
    k = s * w
    W = 2 ** w
    r = 2 ** k
    n_p = mulinv(r-n, r)
    n0 = n_p % W
    aT = radix(a, W)[-s:][::-1]
    bT = radix(b, W)[-s:][::-1]
    nT = radix(n, W)[-s:][::-1]

    t = [0]*(s+2)
    # Steps 1 and 2
    for i in range(s):
        C = 0
        for j in range(s):
            d_w = t[j] + aT[j]*bT[i] + C
            C, S = d_w // W, d_w % W
            t[j] = S
        d_w = t[s] + C
        C, S = d_w // W, d_w % W
        t[s] = S
        t[s + 1] = C
        C = 0
        m = (t[0] * n0) % W
        for j in range(s):
            d_w = t[j] + m * nT[j] + C
            C, S = d_w // W, d_w % W
            t[j] = S
        d_w = t[s] + C
        C, S = d_w // W, d_w % W
        t[s] = S
        t[s + 1] = t[s + 1] + C
        for j in range(s+1):
            t[j] = t[j+1]
    u = t

    # Step 3
    B = 0
    for i in range(s):
        d_w = u[i] - nT[i] - B
        B, D = d_w // W, d_w % W
        t[i] = D
    d_w = u[s] - B
    B, D = d_w // W, d_w % W
    t[s] = D
    if B:
        cT = t[:s]
    else:
        cT = u[:s]

    # Compute value of result
    c = 0
    for i, w in enumerate(cT):
        c += w * W ** i
    return c

    


    
if __name__ == '__main__':
    a = 100
    b = 240
    s = 4
    w = 4
    n = 33533
    c = MonProSOS(a, b, s, w, n)
    c = MonProCIOS(a, b, s, w, n)
\end{lstlisting} 


\section*{Podziękowanie}

Jan Jurec pragnie podziękować swojej narzeczonej Katarzynie za nieustanne wsparcie w uprawianiu nauki, pomoc w ciężkich chwilach i pyszne obiady, które dostarczają mu energii i motywacji do zgłębiania tajników organizacji i arytmetyki komputerów.
Filip Toruń natomiast dziękuje Dr Żołnierkowi, za to, że zaszczepił w nim pasję poznawania i niechęć do prostych rozwiązań.
Obaj zaś dziękują Doktorowi Piotrowi Patronikowi, który dzielił się swoją wiedzą i skutecznie motywował do udanego przeprowadzenia projektu.

\begin{thebibliography}{1}

\bibitem{IEEEhowto:Koc}
C.~K.~Koc and T.~Acar, \emph{Analyzing and Comparing Montgomery Multiplication Algorithms}, IEEE Micro, 16(3):26-33, June 1996.

\end{thebibliography}
\begin{IEEEbiography}[{\includegraphics[width=1in,height=1.25in,clip,keepaspectratio]{jjjurec}}]{Jan Jakub Jurec} jest po raz trzeci studentem trzeciego roku Politechniki Wrocławskiej wydziału Elektroniki kierunku Informatyka. Marzeniem autora jest zdanie kursu Organizacja i Architektura Komputerów za trzecim podejściem, jako że kieruje się zasadą "do trzech razy sztuka"! \\
E-mail: jurec@protonmail.com
\end{IEEEbiography}


\begin{IEEEbiography}[{\includegraphics[width=1in,height=1.25in,clip,keepaspectratio]{ftorun}}]{Filip Toruń}
jest studentem Politechniki Wrocławskiej wydziału Elektroniki kierunku Informatyka. Jego zainteresowania to technologie mobilne, obliczenia binarne, atlasy geograficzne.\\
E-mail: 209428@student.pwr.edu.pl
\end{IEEEbiography}

\vfill

\end{document}


